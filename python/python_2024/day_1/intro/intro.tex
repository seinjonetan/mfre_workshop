\documentclass[serif, 9pt, aspectratio=32]{beamer} 
\usetheme{Darmstadt}

\usepackage{appendixnumberbeamer}
\usepackage{booktabs}
\usepackage[scale=2]{ccicons}
\usepackage{pgfplots}
\usepgfplotslibrary{dateplot}
\usepackage{xspace}
\usepackage{tikz}
\usepackage{hyperref}
\usepackage{xcolor}
\usepackage{listings}
\usetikzlibrary{shapes, arrows}
\newcommand{\themename}{\textbf{\textsc{metropolis}}\xspace}

\title{Introduction to Coding}
\date{\today}
\author{Tan Sein Jone}
\institute{University of British Columbia}

\pgfplotsset{compat=1.18}
\setbeamertemplate{footline}[frame number]

\begin{document}

\maketitle

\begin{frame}{Table of contents}
    \setbeamertemplate{section in toc}[sections numbered]
    \tableofcontents[hideallsubsections]
\end{frame}

\section{Functions}

\begin{frame}
    \frametitle{Table of Contents}
    \setbeamertemplate{section in toc}[sections numbered]
    \tableofcontents[currentsection]
\end{frame}

\begin{frame}
    \centering
    \frametitle{Hello World}
    \begin{itemize}
        \setlength{\itemsep}{3em}
        \item The first program that every programmer writes
        \item How to start a Python program
        \item How to print to the console
    \end{itemize}
\end{frame}

\begin{frame}
    \centering
    \frametitle{What is a Function?}
    \begin{itemize}
        \setlength{\itemsep}{3em}
        \item A function is a block of code that performs a specific task
        \item A function can take in parameters
        \item A function can return a value
    \end{itemize}
\end{frame}

\begin{frame}
    \centering
    \frametitle{Why Use Functions?}
    \begin{itemize}
        \setlength{\itemsep}{3em}
        \item Functions make code more modular
        \item Functions make code more readable
        \item Functions make code more reusable
    \end{itemize}
\end{frame}

\begin{frame}
    \centering
    \frametitle{Calling a Function}
    \begin{itemize}
        \setlength{\itemsep}{3em}
        \item A function can ba called within a program/script
        \item A function can also be called using the terminal
    \end{itemize}
\end{frame}

\begin{frame}[fragile]
    \frametitle{In The Terminal}
    \begin{lstlisting}[language=Python]
        python3 print("Hello, World!")
    \end{lstlisting}
\end{frame}

\begin{frame}
    \centering
    \frametitle{Breakdown}
    \begin{itemize}
        \setlength{\itemsep}{3em}
        \item python3: The Python interpreter
        \item print(): The function
        \item "Hello, World!": The argument
        \item Output: This command will print "Hello, World!" to the console
    \end{itemize}
\end{frame}

\begin{frame}
    \centering
    \frametitle{Why Running Code Like This is Not Ideal}
    \begin{itemize}
        \setlength{\itemsep}{3em}
        \item Code is not saved (in other words, it's ephemeral)
        \item Code is not reusable
        \item Code is not readable (everything is in one line)
        \item Let's instead try to run the code in a script
    \end{itemize}
\end{frame}

\begin{frame}[fragile]
    \frametitle{In The Terminal}
    \begin{lstlisting}[language=Python]
        touch first_notebook.ipynb
    \end{lstlisting}
\end{frame}

\begin{frame}
    \centering
    \frametitle{Breakdown}
    \begin{itemize}
        \setlength{\itemsep}{3em}
        \item touch: Command to create a file
        \item first\_notebook.ipynb: The name of the file
        \item Output: This command will create a file called first\_notebook.ipynb
    \end{itemize}
\end{frame}

\begin{frame}
    \centering
    \frametitle{Jupyter Notebooks vs Python Scripts}
    \begin{itemize}
        \setlength{\itemsep}{3em}
        \item They both run Python code
        \item Jupyter Notebooks are more interactive
        \item Jupyter Notebooks are more visual
        \item Jupyter Notebooks are more user-friendly, due to the ability to run code in cells
        \item It makes it easier to debug code
    \end{itemize}
\end{frame}

\begin{frame}
    \centering
    \frametitle{Running Code in a Jupyter Notebook}
    \begin{itemize}
        \setlength{\itemsep}{3em}
        \item Open the Jupyter Notebook
        \item Create a new cell
        \item Write the code in the cell
        \item Run the cell
    \end{itemize}
\end{frame}

\begin{frame}[fragile]
    \frametitle{first\_notebook.ipynb}
    \begin{lstlisting}[language=Python]
        print("Hello, World!")
    \end{lstlisting}
\end{frame}

\begin{frame}
    \centering
    \frametitle{Functions and Packages/Libraries}
    \begin{itemize}
        \setlength{\itemsep}{3em}
        \item Functions can be defined in packages/libraries
        \item Functions can be imported from packages/libraries
        \item Functions can also be built into the Python interpreter
        \item Certain functions and classes are reserved words
    \end{itemize}
\end{frame}

\begin{frame}
    \centering
    \frametitle{Common Built-in Functions}
    \begin{itemize}
        \setlength{\itemsep}{3em}
        \item print(): Prints to the console
        \item input(): Takes user input
        \item len(): Returns the length of an object
        \item range(): Returns a sequence of numbers
        \item type(): Returns the type of an object
    \end{itemize}
\end{frame}

\begin{frame}
    \centering
    \frametitle{Expressive Languages}
    \begin{itemize}
        \setlength{\itemsep}{3em}
        \item Python is what's known as an expressive language
        \item Expressive languages are designed to be easy to read and write
        \item These languages translate code into machine code
        \item An example of machine code is binary
    \end{itemize}
\end{frame}

\section{Variables}

\begin{frame}
    \frametitle{Table of Contents}
    \setbeamertemplate{section in toc}[sections numbered]
    \tableofcontents[currentsection]
\end{frame}

\begin{frame}
    \centering
    \frametitle{What are Variables?}
    \begin{itemize}
        \setlength{\itemsep}{3em}
        \item Variables are used to store data
        \item Variables are assigned a value
        \item Variables can be changed
    \end{itemize}
\end{frame}

\begin{frame}
    \centering
    \frametitle{Variable Naming Rules}
    \begin{itemize}
        \setlength{\itemsep}{3em}
        \item Variables must start with a letter or underscore
        \item Variables can only contain letters, numbers, and underscores
        \item Variables are case-sensitive
        \item Variables cannot be reserved words
    \end{itemize}
\end{frame}

\begin{frame}
    \centering
    \frametitle{Variable Naming Conventions}
    \begin{itemize}
        \setlength{\itemsep}{3em}
        \item Camel Case: myVariableName
        \item Pascal Case: MyVariableName
        \item Snake Case: my\_variable\_name
    \end{itemize}
\end{frame}

\begin{frame}
    \centering
    \frametitle{Variable Naming Conventions}
    \begin{itemize}
        \setlength{\itemsep}{3em}
        \item Each language has its own naming conventions
        \item Python uses snake case
        \item JavaScript uses camel case
        \item C\# uses pascal case
    \end{itemize}
\end{frame}

\begin{frame}
    \centering
    \frametitle{Variable Naming Tips}
    \begin{itemize}
        \setlength{\itemsep}{3em}
        \item Try to name variables descriptively
        \item But don't make it so descriptive that it's long and hard to read
        \item Bear in mind YOU will have to be the one typing these variables out
        \item Make sure the variable name is relevant to the data it's storing
    \end{itemize}
\end{frame}

\begin{frame}
    \centering
    \frametitle{Good and Bad Examples of Variable Names}
    \begin{itemize}
        \setlength{\itemsep}{3em}
        \item Good: name, age, grade
        \item Too short and ambiguous: n, a, g
        \item Too long and descriptive: name\_of\_student, age\_of\_student, grade\_of\_student
        \item Too long and irrelevant: name\_of\_student\_in\_class, age\_of\_student\_in\_class, grade\_of\_student\_in\_class
    \end{itemize}
\end{frame}

\begin{frame}
    \centering
    \frametitle{Variables and Modern IDEs}
    \begin{itemize}
        \setlength{\itemsep}{3em}
        \item Modern IDEs have features that help with variable naming and autocompletion
        \item Most IDEs have a feature similar to intellesense in Visual Studio Code
        \item This feature will suggest variable names as you type
        \item It will also bring up a list of variables that have also been created
    \end{itemize}
\end{frame}

\begin{frame}
    \centering
    \frametitle{Pay Attention to the Warnings Your IDE Gives You}
    \begin{itemize}
        \setlength{\itemsep}{3em}
        \item IDEs will give you warnings if you use a variable that hasn't been declared
        \item IDEs will give you warnings if you use a variable that has already been declared
        \item IDEs will give you warnings if you use a variable that is not being used
    \end{itemize}
\end{frame}

\begin{frame}
    \centering
    \frametitle{Scope}
    \begin{itemize}
        \setlength{\itemsep}{3em}
        \item Global Variables: Variables declared outside of a function
              \begin{itemize}
                  \item Can be accessed anywhere
              \end{itemize}
        \item Local Variables: Variables declared inside of a function
              \begin{itemize}
                  \item Can only be accessed within the function
              \end{itemize}
    \end{itemize}
\end{frame}

\begin{frame}[fragile]
    \begin{lstlisting}[language=Python]
        def my_function()\:
            x = 10
        x = 20
        my_function()
        print(x)
    \end{lstlisting}
\end{frame}

\begin{frame}
    \centering
    \frametitle{Breakdown}
    \begin{itemize}
        \setlength{\itemsep}{3em}
        \item x = 20: This is a global variable
        \item x = 10: This is a local variable
        \item print(x): This will print 20
        \item This is because the x in the function is a local variable
        \item The x outside of the function is a global variable
    \end{itemize}
\end{frame}

\begin{frame}
    \centering
    \frametitle{Reusing Variable Names}
    \begin{itemize}
        \setlength{\itemsep}{3em}
        \item In general, it's best to avoid reusing variable names, especially if they're part of different scopes
        \item This can lead to confusion
        \item This can lead to errors
        \item You can however get away with this if you import functions from different scripts
    \end{itemize}
\end{frame}

\section{Data Types}

\begin{frame}
    \frametitle{Table of Contents}
    \setbeamertemplate{section in toc}[sections numbered]
    \tableofcontents[currentsection]
\end{frame}

\begin{frame}
    \centering
    \frametitle{Data Types}
    \begin{itemize}
        \setlength{\itemsep}{1em}
        \item Integers: Whole numbers
        \item Floats: Numbers with decimals
        \item Strings: Text
        \item Booleans: True or False
        \item Lists: Ordered collection of items
        \item Tuples: Ordered collection of items that cannot be changed
        \item Dictionaries: Unordered collection of items
        \item Sets: Unordered collection of unique items
    \end{itemize}
\end{frame}

\begin{frame}
    \centering
    \frametitle{Type Checking}
    \begin{itemize}
        \setlength{\itemsep}{3em}
        \item Type checking is used to determine the data type of a variable
        \item Type checking is used to ensure that the correct data type is being used
        \item Type checking is used to prevent errors
    \end{itemize}
\end{frame}

\begin{frame}
    \centering
    \frametitle{Importance of Type Checking}
    \begin{itemize}
        \setlength{\itemsep}{3em}
        \item There are advantages and disadvantages to using each data type
        \item Interacting with different data types can cause errors
    \end{itemize}
\end{frame}

\begin{frame}
    \centering
    \frametitle{Common Data Type Errors}
    \begin{itemize}
        \setlength{\itemsep}{3em}
        \item Mixing data types
        \item Using the wrong data type
        \item Not converting data types
    \end{itemize}
\end{frame}

\begin{frame}
    \centering
    \frametitle{Why Not Store An Int as a String?}
    \begin{itemize}
        \setlength{\itemsep}{3em}
        \item It's not efficient
        \item An int takes up less memory than a string
        \item You cannot perform mathematical operations on a string
    \end{itemize}
\end{frame}

\begin{frame}[fragile]
    \begin{lstlisting}[language=Python]
        random_int = 10
        random_float = 10.0
        random_string = "10"
    \end{lstlisting}
\end{frame}

\begin{frame}
    \centering
    \frametitle{Type Casting}
    \begin{itemize}
        \setlength{\itemsep}{3em}
        \item Converting between data types
        \item Can be done using built-in functions
        \item Not all conversions are possible
    \end{itemize}
\end{frame}

\begin{frame}
    \centering
    \frametitle{Common Type Casting Errors}
    \begin{itemize}
        \setlength{\itemsep}{3em}
        \item Converting a string to an int that is not a number
        \item Converting a float to an int that is not a whole number
        \item Converting a float to an int that is too large
    \end{itemize}
\end{frame}

\begin{frame}
    \centering
    \frametitle{Common Data Type Conversions}
    \begin{itemize}
        \setlength{\itemsep}{3em}
        \item int(): Converts a value to an integer
        \item float(): Converts a value to a float
        \item str(): Converts a value to a string
        \item bool(): Converts a value to a boolean
    \end{itemize}
\end{frame}

\begin{frame}[fragile]
    \begin{lstlisting}[language=Python]
        converted_int = int(random_float)
    \end{lstlisting}
\end{frame}

\begin{frame}
    \centering
    \frametitle{Data Types with Multiple Values}
    \begin{itemize}
        \setlength{\itemsep}{3em}
        \item Lists, Tuples, Dictionaries, and Sets can store multiple values
        \item Each value can be a different data type
        \item Each value can be accessed using an index
        \item Each value can be changed
    \end{itemize}
\end{frame}

\begin{frame}
    \centering
    \frametitle{Common Data Types with Multiple Values}
    \begin{itemize}
        \setlength{\itemsep}{3em}
        \item Lists: Ordered collection of items
        \item Tuples: Ordered collection of items that cannot be changed
        \item Dictionaries: Unordered collection of items
        \item Sets: Unordered collection of unique items
    \end{itemize}
\end{frame}

\begin{frame}[fragile]
    \begin{lstlisting}[language=Python]
        random_list = [10, 10.0, "10"]
        random_tuple = (10, 10.0, "10")
        random_dict = {"int": 10, "float": 10.0, "string": "10"}
        random_set = {10, 10.0, "10"}
    \end{lstlisting}
\end{frame}

\begin{frame}
    \centering
    \frametitle{Common Usecases}
    \begin{itemize}
        \setlength{\itemsep}{3em}
        \item Lists and dicitonaries will likely be the most used data types
        \item Lists are used to store multiple values
        \item Dictionaries are used to store key-value pairs
    \end{itemize}
\end{frame}

\begin{frame}
    \centering
    \frametitle{When to Use a List}
    \begin{itemize}
        \setlength{\itemsep}{3em}
        \item When you need to store multiple values
        \item You need to change the values
        \item You have to keep track of the order of the values
    \end{itemize}
\end{frame}

\begin{frame}
    \centering
    \frametitle{When to Use a Dictionary}
    \begin{itemize}
        \setlength{\itemsep}{3em}
        \item When you need to store key-value pairs
        \item You need to change the values
        \item You don't need to keep track of the order of the values
    \end{itemize}
\end{frame}

\begin{frame}
    \centering
    \frametitle{Common Methods for Lists}
    \begin{itemize}
        \setlength{\itemsep}{3em}
        \item append(): Adds an element to the end of the list
        \item insert(): Adds an element at a specific index
        \item remove(): Removes an element from the list
        \item pop(): Removes an element at a specific index
        \item clear(): Removes all elements from the list
    \end{itemize}
\end{frame}

\begin{frame}
    \centering
    \frametitle{Common Methods for Dictionaries}
    \begin{itemize}
        \setlength{\itemsep}{3em}
        \item get(): Gets the value of a key
        \item keys(): Gets all the keys
        \item values(): Gets all the values
        \item items(): Gets all the key-value pairs
        \item clear(): Removes all key-value pairs
    \end{itemize}
\end{frame}

\begin{frame}
    \centering
    \frametitle{Why Don't We ALways Use Dictionaries?}
    \begin{itemize}
        \setlength{\itemsep}{3em}
        \item Dictionaries are not ordered
        \item Dictionaries are not indexed
        \item Dictionaries are not iterable
    \end{itemize}
\end{frame}

\begin{frame}
    \centering
    \frametitle{Combining Data Types}
    \begin{itemize}
        \setlength{\itemsep}{3em}
        \item Data types can be combined
        \item Lists can store dictionaries
        \item Dictionaries can store lists
    \end{itemize}
\end{frame}

\begin{frame}
    \centering
    \frametitle{Lists of dictionaries}
    \begin{itemize}
        \setlength{\itemsep}{3em}
        \item Say for example we have a list of students
        \item Each student has a name, age, and grade
        \item We can store this information in a list of dictionaries
        \item Each dictionary will represent a student
    \end{itemize}
\end{frame}

\begin{frame}[fragile]
    \begin{lstlisting}
        students = [
            {"name": "Alice", "age": 20, "grade": 90},
            {"name": "Bob", "age": 21, "grade": 85},
            {"name": "Charlie", "age": 22, "grade": 80}
        ]
    \end{lstlisting}
\end{frame}

\begin{frame}
    \centering
    \frametitle{Dictionaries of lists}
    \begin{itemize}
        \setlength{\itemsep}{3em}
        \item Say for example we have a dictionary of students
        \item Each student has a list of grades
        \item We can store this information in a dictionary of lists
        \item Each key will represent a student
    \end{itemize}
\end{frame}

\begin{frame}[fragile]
    \begin{lstlisting}
        students = {
            "Alice": [90, 85, 80],
            "Bob": [85, 80, 75],
            "Charlie": [80, 75, 70]
        }
    \end{lstlisting}
\end{frame}

\begin{frame}
    \centering
    \frametitle{Possible Problem}
    \begin{itemize}
        \setlength{\itemsep}{3em}
        \item Assuming that the order of the list of grades is important
        \item We cannot guarantee that the order of the list of grades will be maintained
        \item This is because dictionaries are not ordered
    \end{itemize}
\end{frame}

\begin{frame}
    \centering
    \frametitle{Possible Solution}
    \begin{itemize}
        \setlength{\itemsep}{3em}
        \item We can use a dictionary of dictionaries instead
        \item Each key will represent a student
        \item Each value will be a dictionary of grades
    \end{itemize}
\end{frame}

\section{Conditionals}

\begin{frame}
    \frametitle{Table of Contents}
    \setbeamertemplate{section in toc}[sections numbered]
    \tableofcontents[currentsection]
\end{frame}

\begin{frame}
    \centering
    \frametitle{What are Conditionals?}
    \begin{itemize}
        \setlength{\itemsep}{3em}
        \item Conditionals are used to make decisions
        \item Conditionals are used to execute code based on a condition
        \item Conditionals are used to compare values
    \end{itemize}
\end{frame}

\begin{frame}
    \centering
    \frametitle{Comparison Operators}
    \begin{itemize}
        \setlength{\itemsep}{3em}
        \item ==: Equal to
        \item !=: Not equal to
        \item \textless: Less than
        \item \textgreater: Greater than
        \item \textless=: Less than or equal to
        \item \textgreater=: Greater than or equal to
    \end{itemize}
\end{frame}

\begin{frame}
    \centering
    \frametitle{Common Use Cases}
    \begin{itemize}
        \setlength{\itemsep}{3em}
        \item If a student's grade is greater than 90, print "A"
        \item If condition is true, execute code
        \item If condition is false, execute other code
    \end{itemize}
\end{frame}

\begin{frame}
    \centering
    \frametitle{Logical Operators}
    \begin{itemize}
        \setlength{\itemsep}{3em}
        \item and: Returns True if both statements are true
        \item or: Returns True if one of the statements is true
        \item not: Returns True if the statement is false
    \end{itemize}
\end{frame}

\begin{frame}
    \centering
    \frametitle{If Statements}
    \begin{itemize}
        \setlength{\itemsep}{3em}
        \item If statements are used to execute code if a condition is true
        \item If statements can be followed by an else statement
        \item If statements can be followed by an elif statement
    \end{itemize}
\end{frame}

\begin{frame}[fragile]
    \begin{lstlisting}[language=Python]
        x = 10
        if x == 10:
            print("x is 10")
        elif x == 20:
            print("x is 20")
        else:
            print("x is not 10 or 20")
    \end{lstlisting}
\end{frame}

\begin{frame}
    \centering
    \frametitle{Breakdown}
    \begin{itemize}
        \setlength{\itemsep}{3em}
        \item x = 10: This is the value of x
        \item if x == 10: This is the condition
        \item print("x is 10"): This is the code that will be executed if the condition is true
        \item elif x == 20: This is the condition that will be checked if the first condition is false
        \item print("x is 20"): This is the code that will be executed if the condition is true
        \item else: This is the code that will be executed if all the conditions are false
    \end{itemize}
\end{frame}

\begin{frame}
    \centering
    \frametitle{Common Mistakes}
    \begin{itemize}
        \setlength{\itemsep}{3em}
        \item Not using the correct comparison operator
        \item Not using the correct logical operator
        \item Not using the correct indentation
    \end{itemize}
\end{frame}

\begin{frame}
    \centering
    \frametitle{Nested If Statements}
    \begin{itemize}
        \setlength{\itemsep}{3em}
        \item If statements can be nested
        \item Nested if statements are used to check multiple conditions
        \item Nested if statements can be difficult to read
    \end{itemize}
\end{frame}

\begin{frame}
    \centering
    \frametitle{Alternatives to Nested If Statements}
    \begin{itemize}
        \setlength{\itemsep}{3em}
        \item Using elif statements
        \item Using logical operators
        \item Using functions
    \end{itemize}
\end{frame}

\section{Loops}

\begin{frame}
    \frametitle{Table of Contents}
    \setbeamertemplate{section in toc}[sections numbered]
    \tableofcontents[currentsection]
\end{frame}

\begin{frame}
    \centering
    \frametitle{What are Loops?}
    \begin{itemize}
        \setlength{\itemsep}{3em}
        \item Loops are used to repeat code
        \item Loops are used to iterate over a sequence
        \item Loops are used to execute code a specific number of times
    \end{itemize}
\end{frame}

\begin{frame}
    \centering
    \frametitle{Common Types of Loops}
    \begin{itemize}
        \setlength{\itemsep}{3em}
        \item For Loops: Used to iterate over a sequence
        \item While Loops: Used to execute code as long as a condition is true
    \end{itemize}
\end{frame}

\begin{frame}
    \centering
    \frametitle{For Loops}
    \begin{itemize}
        \setlength{\itemsep}{3em}
        \item For loops are used to iterate over a sequence
        \item For loops are used to execute code a specific number of times
        \item For loops can be used with lists, tuples, dictionaries, and sets
    \end{itemize}
\end{frame}

\begin{frame}[fragile]
    \begin{lstlisting}[language=Python]
        for x in range(10):
            print(x)
    \end{lstlisting}
\end{frame}

\begin{frame}
    \centering
    \frametitle{Breakdown}
    \begin{itemize}
        \setlength{\itemsep}{3em}
        \item for x in range(10): This is the loop
        \item print(x): This is the code that will be executed
        \item x: This is the variable that will be used to iterate over the sequence
        \item range(10): This is the sequence
    \end{itemize}
\end{frame}

\begin{frame}
    \centering
    \frametitle{While Loops}
    \begin{itemize}
        \setlength{\itemsep}{3em}
        \item While loops are used to execute code as long as a condition is true
        \item While loops are used to execute code a specific number of times
        \item While loops can be used with lists, tuples, dictionaries, and sets
    \end{itemize}
\end{frame}

\begin{frame}[fragile]
    \begin{lstlisting}[language=Python]
        x = 0
        while x < 10:
            print(x)
            x += 1
    \end{lstlisting}
\end{frame}

\begin{frame}
    \centering
    \frametitle{Breakdown}
    \begin{itemize}
        \setlength{\itemsep}{3em}
        \item x = 0: This is the variable that will be used to check the condition
        \item while x < 10: This is the loop
        \item print(x): This is the code that will be executed
        \item x += 1: This is the code that will increment the variable
    \end{itemize}
\end{frame}

\begin{frame}
    \centering
    \frametitle{Common Mistakes}
    \begin{itemize}
        \setlength{\itemsep}{3em}
        \item Not using the correct comparison operator
        \item Not using the correct logical operator
        \item Not using the correct indentation
    \end{itemize}
\end{frame}

\begin{frame}
    \centering
    \frametitle{Nested Loops}
    \begin{itemize}
        \setlength{\itemsep}{3em}
        \item Loops can be nested
        \item Nested loops are used to iterate over multiple sequences
        \item Nested loops can be difficult to read
    \end{itemize}
\end{frame}

\end{document}
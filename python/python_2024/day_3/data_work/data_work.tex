\documentclass[serif, 9pt, aspectratio=32]{beamer} 
\usetheme{Darmstadt}

\usepackage{appendixnumberbeamer}
\usepackage{booktabs}
\usepackage[scale=2]{ccicons}
\usepackage{pgfplots}
\usepgfplotslibrary{dateplot}
\usepackage{xspace}
\usepackage{tikz}
\usepackage{hyperref}
\usepackage{xcolor}
\usepackage{listings}
\usetikzlibrary{shapes, arrows}
\newcommand{\themename}{\textbf{\textsc{metropolis}}\xspace}

\title{Python Applications}
\date{\today}
\author{Tan Sein Jone}
\institute{University of British Columbia}

\pgfplotsset{compat=1.18}
\setbeamertemplate{footline}[frame number]

\begin{document}

\maketitle

\begin{frame}{Table of contents}
    \setbeamertemplate{section in toc}[sections numbered]
    \tableofcontents[hideallsubsections]
\end{frame}

\section{Data Preview}

\begin{frame}
    \frametitle{Table of Contents}
    \setbeamertemplate{section in toc}[sections numbered]
    \tableofcontents[currentsection]
\end{frame}

\begin{frame}
    \centering
    \frametitle{Pollution Data}
    \begin{itemize}
        \setlength{\itemsep}{2em}
        \item Synthetic data on pollution levels in a province by industry and year
        \item Data is in a csv file
        \item Relatively long dataset
    \end{itemize}
\end{frame}

\section{Data Wrangler}

\begin{frame}
    \frametitle{Table of Contents}
    \setbeamertemplate{section in toc}[sections numbered]
    \tableofcontents[currentsection]
\end{frame}

\begin{frame}{Data Wrangler}
    \begin{itemize}
        \setlength{\itemsep}{3em}
        \item Data Wrangler is an extension in VS Code that allows you to preview data in a more readable format.
        \item It is useful for viewing data in a tabular format.
        \item It provides important summary statistics for the data.
    \end{itemize}
\end{frame}

\begin{frame}{Data Wrangler}
    \begin{itemize}
        \setlength{\itemsep}{3em}
        \item You can access it in a couple of ways in VS Code.
        \item Open a csv file and in the top right corner, click on the icon that looks like a table.
        \item You can also right-click on a csv file and select "Show in Data Wrangler".
    \end{itemize}
\end{frame}

\begin{frame}
    \begin{itemize}
        \setlength{\itemsep}{3em}
        \item When you load in data to the python kernel in VS Code, you can also view the data in Data Wrangler.
        \item In the bottom tool bar, you should see a tab for Jupyter.
        \item You should be able to see a button to the left of each variable that allows you to view the data in Data Wrangler.
    \end{itemize}
\end{frame}

\section{Data Manipulation}

\begin{frame}
    \frametitle{Table of Contents}
    \setbeamertemplate{section in toc}[sections numbered]
    \tableofcontents[currentsection]
\end{frame}

\begin{frame}
    \centering
    \frametitle{Data Manipulation}
    \begin{itemize}
        \setlength{\itemsep}{2em}
        \item Select columns and rows
        \item Get pollution numbers for BC
    \end{itemize}
\end{frame}

\begin{frame}
    \centering
    \frametitle{Data Manipulation}
    \begin{itemize}
        \setlength{\itemsep}{2em}
        \item Filter data
        \item Filter out pollution data post 2000
        \item Name the highest polluting sector for every province in every year
    \end{itemize}
\end{frame}

\begin{frame}
    \centering
    \frametitle{Data Manipulation}
    \begin{itemize}
        \setlength{\itemsep}{2em}
        \item Filter and Group data
        \item Total pollution for every province sector pair
        \item Get total pollution by province
        \item Group pollution into pre 2000 and post 2000
    \end{itemize}
\end{frame}

\begin{frame}
    \centering
    \frametitle{Data Manipulation}
    \begin{itemize}
        \setlength{\itemsep}{2em}
        \item Merge data
        \item Map full names to province abbreviations
    \end{itemize}
\end{frame}

\begin{frame}
    \centering
    \frametitle{Data Manipulation}
    \begin{itemize}
        \setlength{\itemsep}{2em}
        \item Applying functions
        \item Assume you want to change pollution values to logs for all columns except 'Year' and 'Province'
        \item Getting Total pollution for every sector
        \item Get a summary of pollution for every province
    \end{itemize}
\end{frame}

\section{Cleaning}

\begin{frame}
    \frametitle{Table of Contents}
    \setbeamertemplate{section in toc}[sections numbered]
    \tableofcontents[currentsection]
\end{frame}

\begin{frame}
    \centering
    \frametitle{Cleaning}
    \begin{itemize}
        \setlength{\itemsep}{2em}
        \item Handling duplicates
        \item Handling missing values
    \end{itemize}
\end{frame}

\section{Visualization}

\begin{frame}
    \frametitle{Table of Contents}
    \setbeamertemplate{section in toc}[sections numbered]
    \tableofcontents[currentsection]
\end{frame}

\begin{frame}
    \centering
    \frametitle{Visualization}
    \begin{itemize}
        \setlength{\itemsep}{2em}
        \item Using matplotlib
        \item Plot a line graph of pollution by sector for British Columbia over time
    \end{itemize}
\end{frame}

\begin{frame}
    \centering
    \frametitle{Visualization}
    \begin{itemize}
        \setlength{\itemsep}{2em}
        \item Using seaborn
        \item Total emissions over time by province
    \end{itemize}
\end{frame}

\begin{frame}
    \centering
    \frametitle{Visualization}
    \begin{itemize}
        \setlength{\itemsep}{2em}
        \item Heatmap
    \end{itemize}
\end{frame}

\section{Task}

\begin{frame}
    \frametitle{Table of Contents}
    \setbeamertemplate{section in toc}[sections numbered]
    \tableofcontents[currentsection]
\end{frame}

\begin{frame}
    \centering
    \frametitle{Task}
    \begin{itemize}
        \setlength{\itemsep}{2em}
        \item Plot emissions over time by sector
        \item Track the total pollution of the top 5 polluters in the year 2000
    \end{itemize}
\end{frame}

\end{document}
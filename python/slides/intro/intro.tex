\documentclass[serif, 9pt, aspectratio=32]{beamer} 
\usetheme{Darmstadt}

\usepackage{appendixnumberbeamer}
\usepackage{booktabs}
\usepackage[scale=2]{ccicons}
\usepackage{pgfplots}
\usepgfplotslibrary{dateplot}
\usepackage{xspace}
\usepackage{tikz}
\usepackage{hyperref}
\usepackage{xcolor}
\usepackage{listings}
\usetikzlibrary{shapes, arrows}
\newcommand{\themename}{\textbf{\textsc{metropolis}}\xspace}

\title{Introduction to Coding}
\date{\today}
\author{Tan Sein Jone}
\institute{University of British Columbia}

\pgfplotsset{compat=1.18}
\setbeamertemplate{footline}[frame number]

\begin{document}

\maketitle

\begin{frame}{Table of contents}
    \setbeamertemplate{section in toc}[sections numbered]
    \tableofcontents[hideallsubsections]
\end{frame}

\section{Functions}

\begin{frame}
    \frametitle{Table of Contents}
    \setbeamertemplate{section in toc}[sections numbered]
    \tableofcontents[currentsection]
\end{frame}

\begin{frame}
    \centering
    \frametitle{Hello World}
    \begin{itemize}
        \setlength{\itemsep}{3em}
        \item The first program that every programmer writes
        \item How to start a Python program
        \item How to print to the console
    \end{itemize}
\end{frame}

\begin{frame}[fragile]
    \frametitle{In Terminal}
    \begin{lstlisting}[language=Python]
        code hello_world.py
    \end{lstlisting}
\end{frame}

\begin{frame}[fragile]
    \frametitle{hello\_world.py}
    \begin{lstlisting}[language=Python]
        print("Hello, World!")
    \end{lstlisting}
\end{frame}

\section{Variables}

\begin{frame}
    \frametitle{Table of Contents}
    \setbeamertemplate{section in toc}[sections numbered]
    \tableofcontents[currentsection]
\end{frame}

\begin{frame}
    \centering
    \frametitle{What are Variables?}
    \begin{itemize}
        \setlength{\itemsep}{3em}
        \item Variables are used to store data
        \item Variables are assigned a value
        \item Variables can be changed
    \end{itemize}
\end{frame}

\begin{frame}
    \centering
    \frametitle{Variable Naming Rules}
    \begin{itemize}
        \setlength{\itemsep}{3em}
        \item Variables must start with a letter or underscore
        \item Variables can only contain letters, numbers, and underscores
        \item Variables are case-sensitive
        \item Variables cannot be reserved words
    \end{itemize}
\end{frame}

\begin{frame}
    \centering
    \frametitle{Variable Naming Conventions}
    \begin{itemize}
        \setlength{\itemsep}{3em}
        \item Camel Case: myVariableName
        \item Pascal Case: MyVariableName
        \item Snake Case: my\_variable\_name
    \end{itemize}
\end{frame}

\begin{frame}
    \centering
    \frametitle{Scope}
    \begin{itemize}
        \setlength{\itemsep}{3em}
        \item Global Variables: Variables declared outside of a function
              \begin{itemize}
                  \item Can be accessed anywhere
              \end{itemize}
        \item Local Variables: Variables declared inside of a function
              \begin{itemize}
                  \item Can only be accessed within the function
              \end{itemize}
    \end{itemize}
\end{frame}

\begin{frame}[fragile]
    \begin{lstlisting}[language=Python]
        def my_function()\:
            x = 10
        x = 20
        my_function()
        print(x)
    \end{lstlisting}
\end{frame}

\section{Data Types}

\begin{frame}
    \frametitle{Table of Contents}
    \setbeamertemplate{section in toc}[sections numbered]
    \tableofcontents[currentsection]
\end{frame}

\begin{frame}
    \centering
    \frametitle{Data Types}
    \begin{itemize}
        \setlength{\itemsep}{1em}
        \item Integers: Whole numbers
        \item Floats: Numbers with decimals
        \item Strings: Text
        \item Booleans: True or False
        \item Lists: Ordered collection of items
        \item Tuples: Ordered collection of items that cannot be changed
        \item Dictionaries: Unordered collection of items
        \item Sets: Unordered collection of unique items
    \end{itemize}
\end{frame}

\begin{frame}
    \centering
    \frametitle{Importance of Type Checking}
    \begin{itemize}
        \setlength{\itemsep}{3em}
        \item There are advantages and disadvantages to using each data type
        \item Interacting with different data types can cause errors
    \end{itemize}
\end{frame}

\begin{frame}[fragile]
    \begin{lstlisting}[language=Python]
        random_int = 10
        random_float = 10.0
        random_string = "10"
    \end{lstlisting}
\end{frame}

\begin{frame}
    \centering
    \frametitle{Type Casting}
    \begin{itemize}
        \setlength{\itemsep}{3em}
        \item Converting between data types
        \item Can be done using built-in functions
        \item
    \end{itemize}
\end{frame}

\begin{frame}[fragile]
    \begin{lstlisting}[language=Python]
        converted_int = int(random_float)
    \end{lstlisting}

\end{frame}

\end{document}
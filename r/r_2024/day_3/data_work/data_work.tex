\documentclass[serif, 9pt, aspectratio=32]{beamer} 
\usetheme{Darmstadt}

\usepackage{appendixnumberbeamer}
\usepackage{booktabs}
\usepackage[scale=2]{ccicons}
\usepackage{pgfplots}
\usepgfplotslibrary{dateplot}
\usepackage{xspace}
\usepackage{tikz}
\usepackage{hyperref}
\usepackage{xcolor}
\usepackage{listings}
\usetikzlibrary{shapes, arrows}
\newcommand{\themename}{\textbf{\textsc{metropolis}}\xspace}

\title{Python Applications}
\date{\today}
\author{Tan Sein Jone}
\institute{University of British Columbia}

\pgfplotsset{compat=1.18}
\setbeamertemplate{footline}[frame number]

\begin{document}

\maketitle

\begin{frame}{Table of contents}
    \setbeamertemplate{section in toc}[sections numbered]
    \tableofcontents[hideallsubsections]
\end{frame}

\section{Data Preview}

\begin{frame}
    \frametitle{Table of Contents}
    \setbeamertemplate{section in toc}[sections numbered]
    \tableofcontents[currentsection]
\end{frame}

\begin{frame}
    \centering
    \frametitle{Pollution Data}
    \begin{itemize}
        \setlength{\itemsep}{2em}
        \item Synthetic data on pollution levels in a province by industry and year
        \item Data is in a csv file
        \item Relatively long dataset
    \end{itemize}
\end{frame}

\section{Variable Preview}

\begin{frame}
    \frametitle{Table of Contents}
    \setbeamertemplate{section in toc}[sections numbered]
    \tableofcontents[currentsection]
\end{frame}

\begin{frame}
    \centering
    \frametitle{Variable Preview}
    \begin{itemize}
        \setlength{\itemsep}{2em}
        \item In R studio, there's a section in the IDE that allows you to view the variables in your environment
        \item This is useful for checking the data types of your variables
        \item You can also see the first few rows of your data
        \item I don't know any extensions in R studio that has the same functionality as Data Wrangler in VS Code, but if you do let me know!
    \end{itemize}
\end{frame}

\section{Data Manipulation}

\begin{frame}
    \frametitle{Table of Contents}
    \setbeamertemplate{section in toc}[sections numbered]
    \tableofcontents[currentsection]
\end{frame}

\begin{frame}{Data Manipulation}
    \begin{itemize}
        \setlength{\itemsep}{3em}
        \item Data manipulation is the process of changing data to make it easier to read or more organized.
        \item This can involve changing the data type of a variable, removing missing values, or creating new variables.
        \item In R, the dplyr package is commonly used for data manipulation.
    \end{itemize}
\end{frame}

\begin{frame}{Data Manipulation}
    \begin{itemize}
        \setlength{\itemsep}{3em}
        \item Filter data after 2010
        \item Get pollution data for the transport sector
        \item Total pollution by province
        \item Applying functions
    \end{itemize}
\end{frame}

\section{Data Visualization}

\begin{frame}
    \frametitle{Table of Contents}
    \setbeamertemplate{section in toc}[sections numbered]
    \tableofcontents[currentsection]
\end{frame}

\begin{frame}{Data Visualization}
    \begin{itemize}
        \setlength{\itemsep}{3em}
        \item Data visualization is the process of representing data graphically.
        \item This can help you identify patterns in the data that may not be obvious from looking at the raw data.
        \item In R, the ggplot2 package is commonly used for data visualization.
    \end{itemize}
\end{frame}

\begin{frame}{Data Visualization}
    \begin{itemize}
        \setlength{\itemsep}{3em}
        \item Create a plot of pollution levels by province
        \item Create a plot of pollution levels over time
        \item Plot pollution as a Heatmap
    \end{itemize}
\end{frame}

\section{Diff in diff}

\begin{frame}
    \frametitle{Table of Contents}
    \setbeamertemplate{section in toc}[sections numbered]
    \tableofcontents[currentsection]
\end{frame}

\begin{frame}{Diff in diff}
    \begin{itemize}
        \setlength{\itemsep}{3em}
        \item Difference in differences is a statistical technique used to estimate the causal effect of a treatment or intervention.
        \item It compares the change in outcomes over time between a treatment group and a control group.
        \item In R, the plm package is commonly used for difference in differences analysis.
    \end{itemize}
\end{frame}

\begin{frame}{Diff in diff}
    \begin{itemize}
        \setlength{\itemsep}{3em}
        \item Assume there's a policy implemented in BC in 2000
        \item Run a diff-in-diff analysis to estimate the effect of the policy
    \end{itemize}
\end{frame}

\section{Regression Discontinuity Design (RDD)}

\begin{frame}
    \frametitle{Table of Contents}
    \setbeamertemplate{section in toc}[sections numbered]
    \tableofcontents[currentsection]
\end{frame}

\begin{frame}{Regression Discontinuity Design (RDD)}
    \begin{itemize}
        \setlength{\itemsep}{3em}
        \item Regression discontinuity design is a quasi-experimental design used to estimate the causal effect of a treatment or intervention.
        \item It exploits the fact that individuals on either side of a cutoff point are similar in all other respects.
        \item In R, the rdd package is commonly used for regression discontinuity design analysis.
    \end{itemize}
\end{frame}

\begin{frame}{Regression Discontinuity Design (RDD)}
    \begin{itemize}
        \setlength{\itemsep}{3em}
        \item Assume there's a policy implemented in 2000
        \item Run a regression discontinuity design analysis to estimate the effect of the policy
    \end{itemize}
\end{frame}

\section{Tasks}

\begin{frame}
    \frametitle{Table of Contents}
    \setbeamertemplate{section in toc}[sections numbered]
    \tableofcontents[currentsection]
\end{frame}

\begin{frame}{Tasks}
    \begin{itemize}
        \setlength{\itemsep}{3em}
        \item Calculate the total pollution for each province in the year 2015.
        \item Plot the trend of pollution over time for the 'Transport' sector, aggregating all provinces.
        \item Plot the total pollution over time for the top 5 provinces with the highest pollution levels in 1999.
    \end{itemize}
\end{frame}

\end{document}